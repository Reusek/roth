%% ===================================================================
%% SEKCE 2.1: JAZYK FORTH
%% ===================================================================

\subsection{Jazyk Forth}
\label{sec:forth}

Jazyk \Forth{} představuje jedinečnou kategorii programovacích jazyků,
která se fundamentálně odlišuje od většiny běžně používaných jazyků.
Jeho minimalistický design, zásobníková architektura a~interaktivní
povaha z~něj činí ideální základ pro studium principů kompilace
a~návrhu programovacích jazyků.


\subsubsection{Historie a filozofie}
\label{sec:forth-historie}

Jazyk \Forth{} byl vytvořen Charlesem H. Moorem na přelomu šedesátých
a~sedmdesátých let 20. století. Moore původně vyvíjel tento jazyk
pro řízení radioteleskopů na observatoři Kitt Peak v~Arizoně, kde
potřeboval jazyk, který by umožnil interaktivní vývoj a~ladění
v~prostředí s~omezenými výpočetními prostředky \autocite{moore1974}.

Název \Forth{} vznikl jako zkratka pro \emph{fourth generation language}
(jazyk čtvrté generace), přičemž kvůli omezením souborového systému
IBM 1130, který povoloval pouze pětiznaková jména souborů, byl název
zkrácen na FORTH. Moore považoval svůj jazyk za výrazný krok kupředu
oproti tehdejším jazykům třetí generace, jako byly FORTRAN a~COBOL.

Filozofie jazyka \Forth{} je založena na několika klíčových principech,
které Leo Brodie shrnul ve své vlivné knize \emph{Thinking Forth}
\autocite{brodie2004}:

\paragraph{Jednoduchost a minimalismus.}
\Forth{} upřednostňuje jednoduchou implementaci před složitou syntaxí.
Základní jádro jazyka může být implementováno v~řádově stovkách řádků
kódu, což umožňuje jeho nasazení i~na velmi omezených platformách.
Tato jednoduchost není náhodná --- Moore věřil, že složitost je
nepřítelem spolehlivosti a~údržby.

\paragraph{Kompozice malých částí.}
Programy ve \Forth{}u jsou budovány z~malých, opakovaně použitelných
jednotek nazývaných \emph{slova} (words). Každé slovo by mělo řešit
jeden konkrétní problém a~jeho délka by neměla přesáhnout zhruba
jeden řádek. Tento přístup předznamenal moderní principy jako
\emph{Unix philosophy} a~mikroslužby.

\paragraph{Interaktivita a okamžitá zpětná vazba.}
\Forth{} byl od počátku navržen jako interaktivní jazyk. Programátor
může okamžitě testovat jednotlivá slova a~pozorovat jejich efekt
na zásobník. Tato vlastnost činí \Forth{} ideálním nástrojem pro
explorativní programování a~ladění v~reálném čase.

\paragraph{Programátor jako architekt.}
Na rozdíl od jazyků, které definují pevnou strukturu programů,
\Forth{} poskytuje programátorovi nástroje pro vytvoření vlastního
jazyka specifického pro danou doménu. Každý \Forth{} program je
vlastně rozšířením jazyka samotného.

Historicky našel \Forth{} široké uplatnění v~oblastech, kde jsou
kritické efektivita a~přímý přístup k~hardware:

\begin{itemize}
	\item \textbf{Embedded systémy} --- díky malým nárokům na paměť
	      a~možnosti přímé manipulace s~registry.
	\item \textbf{Kosmický průmysl} --- NASA použila \Forth{} v~řadě
	      vesmírných misí včetně sondy Philae.
	\item \textbf{Inicializace systémů} --- standard IEEE 1275 (Open
	      Firmware) používá \Forth{} pro bootovací firmware
	      \autocite{open-firmware}.
	\item \textbf{Radioastronomie} --- původní doména jazyka, kde se
	      používá dodnes.
\end{itemize}


\subsubsection{Zásobníková architektura}
\label{sec:forth-zasobnik}

Fundamentálním rysem jazyka \Forth{} je jeho zásobníková architektura.
Na rozdíl od většiny programovacích jazyků, které používají pojmenované
proměnné pro předávání dat mezi operacemi, \Forth{} využívá implicitní
datový zásobník \autocite{koopman1989}.

\paragraph{Datový zásobník.}
Datový zásobník (data stack, také parameter stack) je primární
strukturou pro předávání hodnot mezi slovy. Operace odebírají
své operandy z~vrcholu zásobníku a~ukládají výsledky zpět na zásobník.
Tento přístup eliminuje potřebu explicitního pojmenování mezivýsledků.

Uvažujme výpočet aritmetického výrazu \texttt{(3 + 4) * 5}.
V~jazyce \Forth{} by tento výpočet vypadal následovně:

\begin{lstlisting}[basicstyle=\small\ttfamily]
3 4 + 5 *
\end{lstlisting}

Průběh výpočtu demonstruje tabulka~\ref{tab:stack-trace}.

\begin{table}[ht]
	\centering
	\caption{Průběh výpočtu výrazu na zásobníku}
	\label{tab:stack-trace}
	\begin{tabular}{lll}
		\toprule
		\textbf{Operace} & \textbf{Zásobník} & \textbf{Popis}              \\
		\midrule
		\texttt{3}       & \texttt{3}        & Uložení konstanty 3         \\
		\texttt{4}       & \texttt{3 4}      & Uložení konstanty 4         \\
		\texttt{+}       & \texttt{7}        & Součet dvou vrchních hodnot \\
		\texttt{5}       & \texttt{7 5}      & Uložení konstanty 5         \\
		\texttt{*}       & \texttt{35}       & Součin dvou vrchních hodnot \\
		\bottomrule
	\end{tabular}
\end{table}

\paragraph{Návratový zásobník.}
Kromě datového zásobníku používá \Forth{} také návratový zásobník
(return stack), který primárně slouží pro uložení návratových adres
při volání slov. Programátor má však přístup i~k~tomuto zásobníku
prostřednictvím slov \texttt{>R} (přesun z~datového na návratový zásobník),
\texttt{R>} (opačný směr) a~\texttt{R@} (kopie vrcholu návratového
zásobníku). Tato vlastnost umožňuje dočasné \uv{odložení} hodnot
z~datového zásobníku.

\paragraph{Reverzní polská notace.}
Zásobníková architektura přirozeně vede k~použití postfixové neboli
reverzní polské notace (Reverse Polish Notation, RPN). V~této notaci
se operátor zapisuje za své operandy, nikoliv mezi ně. RPN má několik
výhod:

\begin{itemize}
	\item Nevyžaduje závorky pro určení priority operací.
	\item Odpovídá přirozenému pořadí vyhodnocování na zásobníku.
	\item Zjednodušuje implementaci parseru a~interpretru.
	\item Vede ke kompaktnějšímu zápisu složitých výrazů.
\end{itemize}

\paragraph{Stack effects.}
Konvence ve \Forth{}u vyžaduje dokumentovat efekt každého slova
na zásobník pomocí notace zvané \emph{stack effect comment}.
Tato notace má tvar \texttt{( before -- after )}, kde \texttt{before}
popisuje hodnoty očekávané na vstupu a~\texttt{after} hodnoty
zanechané na výstupu. Například:

\begin{itemize}
	\item \texttt{+ ( n1 n2 -- n3 )} --- sečte dvě čísla
	\item \texttt{DUP ( n -- n n )} --- zduplikuje vrchol zásobníku
	\item \texttt{SWAP ( n1 n2 -- n2 n1 )} --- prohodí dva vrchní prvky
	\item \texttt{DROP ( n -- )} --- odstraní vrchol zásobníku
\end{itemize}

Tyto anotace jsou klíčové pro správné použití slov, neboť při
nesprávném počtu hodnot na zásobníku dochází k~chybám za běhu.


\subsubsection{Syntaxe a sémantika}
\label{sec:forth-syntax}

Syntaxe jazyka \Forth{} je extrémně minimalistická. Zdrojový kód
se skládá ze \emph{slov} oddělených bílými znaky. Neexistují
žádné speciální oddělovače příkazů, závorky pro volání funkcí
ani klíčová slova v~tradičním smyslu \autocite{brodie1981}.

\paragraph{Slova.}
Základní jednotkou programu ve \Forth{}u je \emph{slovo} (word).
Slovo je sekvence znaků oddělená bílými znaky, která reprezentuje
buď vestavěnou operaci (primitive), uživatelem definovanou funkci,
nebo číselnou konstantu.

\begin{lstlisting}[basicstyle=\small\ttfamily]
( Definice noveho slova KVADRAT )
: KVADRAT ( n -- n^2 )  DUP * ;

( Pouziti )
5 KVADRAT .   ( Vypise: 25 )
\end{lstlisting}

Definice nového slova začíná dvojtečkou (\texttt{:}), následuje
název slova, tělo definice a~ukončení středníkem (\texttt{;}).
Komentáře se uvádějí v~kulatých závorkách.

\paragraph{Čísla.}
Číselné konstanty jsou rozpoznány lexikálním analyzátorem a~při
zpracování jsou automaticky uloženy na zásobník. \Forth{} tradičně
podporuje celočíselnou aritmetiku, přičemž moderní implementace
často přidávají podporu pro čísla s~plovoucí řádovou čárkou.

\paragraph{Řídicí struktury.}
\Forth{} poskytuje základní řídicí struktury pro větvení a~cykly:

\begin{lstlisting}[basicstyle=\small\ttfamily]
( Podminka IF-ELSE-THEN )
: SIGNUM ( n -- -1|0|1 )
  DUP 0 < IF DROP -1 ELSE
  DUP 0 > IF DROP 1 ELSE DROP 0 THEN THEN ;

( Pocitany cyklus DO-LOOP )
: VYPIS-5 ( -- )  5 0 DO I . LOOP ;

( Cyklus s podminkou BEGIN-WHILE-REPEAT )
: COUNTDOWN ( n -- )
  BEGIN DUP 0 > WHILE DUP . 1 - REPEAT DROP ;
\end{lstlisting}

Důležité je si povšimnout, že \texttt{IF} ve \Forth{}u je postfixové ---
podmínka je vyhodnocena před klíčovým slovem \texttt{IF}, které
pak odebere hodnotu ze zásobníku a~rozhodne o~dalším průběhu.

\paragraph{Proměnné a paměť.}
Ačkoliv \Forth{} primárně pracuje se zásobníkem, poskytuje také
mechanismy pro práci s~pojmenovanou pamětí:

\begin{lstlisting}[basicstyle=\small\ttfamily]
VARIABLE POCITADLO      ( Deklarace promenne )
0 POCITADLO !           ( Inicializace na 0 )
POCITADLO @             ( Cteni hodnoty )
POCITADLO @ 1+ POCITADLO !  ( Inkrementace )
\end{lstlisting}

Operátor \texttt{!} (store) ukládá hodnotu na adresu, operátor
\texttt{@} (fetch) čte hodnotu z~adresy. Název proměnné vrací
její adresu, nikoliv hodnotu.


\subsubsection{Existující implementace}
\label{sec:forth-implementace}

Za více než padesát let existence jazyka \Forth{} vzniklo mnoho
jeho implementací, od minimalistických embedded verzí po plnohodnotné
vývojové systémy.

\paragraph{Gforth.}
Gforth (GNU Forth) je referenční implementací standardu
Forth 2012 \autocite{gforth}. Je vyvíjen pod záštitou projektu GNU
a~představuje nejpoužívanější open-source implementaci \Forth{}u.
Gforth je napsán v~jazycích C a~\Forth{} a~je dostupný pro většinu
operačních systémů. Mezi jeho hlavní přednosti patří:

\begin{itemize}
	\item Plná kompatibilita se standardem Forth 2012.
	\item Rozsáhlá dokumentace a~aktivní komunita.
	\item Efektivní interpretace pomocí techniky \emph{threaded code}
	      \autocite{ertl2002}.
	\item Podpora pro čísla s~plovoucí řádovou čárkou a~dynamickou
	      alokaci paměti.
\end{itemize}

\paragraph{Historické a specializované implementace.}
Mezi další významné implementace patří:

\begin{itemize}
	\item \textbf{fig-Forth} --- historická implementace od Forth Interest
	      Group, která pomohla rozšířit jazyk na osmibitové počítače.
	\item \textbf{SwiftForth} --- komerční implementace od společnosti
	      Forth, Inc., optimalizovaná pro výkon.
	\item \textbf{Mecrisp} --- moderní implementace zaměřená na
	      mikrokontroléry ARM Cortex-M.
	\item \textbf{Factor} --- moderní zásobníkový jazyk inspirovaný
	      \Forth{}em, rozšířený o~objektově orientované programování,
	      garbage collection a~bohatou standardní knihovnu.
\end{itemize}

\paragraph{Standard Forth 2012.}
Jazyk \Forth{} byl standardizován americkým institutem ANSI v~roce 1994
(ANSI X3.215-1994) a~později rozšířen komunitním standardem Forth 2012
\autocite{forth-standard}. Standard definuje základní slovníkové sady
(word sets):

\begin{description}
	\item[Core] --- základní slova nezbytná pro každou implementaci
	\item[Core Extensions] --- užitečná rozšíření základní sady
	\item[Block] --- práce s~blokovým souborovým systémem
	\item[Double-Number] --- podpora pro dvojnásobnou přesnost
	\item[Exception] --- mechanismus pro zpracování výjimek
	\item[Facility] --- interakce s~uživatelem a~systémem
	\item[File] --- práce se soubory
	\item[Floating-Point] --- čísla s~plovoucí řádovou čárkou
	\item[Locals] --- lokální proměnné
	\item[Memory-Allocation] --- dynamická alokace paměti
	\item[Programming-Tools] --- ladící nástroje
	\item[Search-Order] --- správa slovníků
	\item[String] --- práce s~řetězci
\end{description}

\paragraph{Vztah jazyka Roth k~existujícím implementacím.}
Jazyk \Roth{}, jehož implementace je předmětem této práce, se inspiruje
základními koncepty jazyka \Forth{}, ale záměrně se nedrží striktně
standardu. Cílem je vytvořit moderní zásobníkový jazyk, který zachovává
filozofii a~eleganci \Forth{}u, ale využívá moderní techniky kompilace
a~optimalizace. Kompilátor \Roth{} generuje nativní kód prostřednictvím
backendů pro jazyky Rust a~C, což jej odlišuje od interpretovaných
implementací jako je Gforth.
