%% ===================================================================
%% KAPITOLA 1: ÚVOD
%% ===================================================================
\section{Úvod}
\label{sec:uvod}

Zásobníkové programovací jazyky představují specifickou kategorii jazyků,
které se od běžných jazyků odlišují fundamentálním přístupem k~předávání
dat mezi operacemi. Místo pojmenovaných proměnných a~explicitního předávání
argumentů využívají implicitní datový zásobník, na který operace ukládají
své výsledky a~ze kterého odebírají své operandy. Tento přístup vede
k~minimalistické syntaxi a~vysoké kompozičnosti kódu.

Nejvýznamnějším představitelem zásobníkových jazyků je \Forth{}, navržený
Charlesem Moorem v~šedesátých letech 20. století. \Forth{} si získal
oblibu zejména v~oblasti vestavěných systémů a~řízení hardware, kde
jeho jednoduchost a~efektivita představují významné výhody. Přestože
dnes \Forth{} nepatří mezi nejrozšířenější programovací jazyky, jeho koncepty
ovlivnily návrh mnoha moderních technologií, včetně virtuálních strojů,
jako jsou JVM a WebAssembly.

Tato práce se zabývá návrhem a~implementací kompilátoru pro zásobníkový
jazyk nazvaný \Roth{}, který je inspirován jazykem \Forth{}, ale přináší
některé moderní prvky. Kompilátor je implementován v~jazyce Rust, což
umožňuje využít jeho silný typový systém a~záruky bezpečnosti paměti
bez nutnosti použití garbage collectoru.


\subsection{Motivace}
\label{sec:motivace}

Motivace pro tuto práci vychází z~několika oblastí:

\paragraph{Vzdělávací hodnota zásobníkových jazyků.}
Zásobníkové jazyky poskytují jedinečný pohled na principy výpočtu
a~kompilace. Jejich jednoduchost umožňuje soustředit se na podstatu
problému bez syntaktického balastu běžných jazyků. Studium a~implementace
takového jazyka nabízí hluboký vhled do fungování kompilátorů, optimalizací
a~generování kódu.

\paragraph{Praktické využití kompilátoru.}
Kompilátor \Roth{} není pouze akademickým cvičením. Generuje efektivní
kód pro reálné platformy (Rust, C) a~obsahuje interaktivní prostředí
REPL s~JIT kompilací, které umožňuje okamžitou zpětnou vazbu při vývoji.
Tato kombinace činí z~\Roth{} nástroj použitelný jak pro výuku, tak
pro experimentování s~nízkoúrovňovým programováním.

\paragraph{Volba jazyka Rust pro implementaci.}
Rust představuje moderní systémový programovací jazyk, který kombinuje
výkon jazyků jako C nebo C++ s~bezpečností paměti garantovanou již
v~době kompilace. Pro implementaci kompilátoru nabízí Rust několik
výhod: algebraické datové typy ideální pro reprezentaci AST a~mezikódu,
pattern matching pro elegantní zpracování syntaktických konstrukcí,
a~ownership systém zabraňující běžným chybám při správě paměti.


\subsection{Cíle práce}
\label{sec:cile}

Hlavním cílem práce je navrhnout a~implementovat plnohodnotný kompilátor
zásobníkového jazyka. Tento cíl lze rozdělit do následujících dílčích cílů:

\begin{enumerate}
	\item \textbf{Návrh jazyka \Roth{}} --- Definovat syntaxi a~sémantiku
	      zásobníkového jazyka inspirovaného jazykem \Forth{}, který bude
	      dostatečně expresivní pro praktické použití.

	\item \textbf{Implementace kompilátoru} --- Vytvořit kompilátor
	      v~jazyce Rust, který provádí lexikální, syntaktickou a~sémantickou
	      analýzu zdrojového kódu a~generuje mezikód (IR).

	\item \textbf{Optimalizace mezikódu} --- Implementovat sadu
	      optimalizačních průchodů včetně constant foldingu, eliminace
	      mrtvého kódu, peephole optimalizací a~inliningu.

	\item \textbf{Generování cílového kódu} --- Vytvořit backendy
	      pro generování kódu v~jazycích Rust a~C, což umožní
	      přenositelnost na různé platformy.

	\item \textbf{REPL s~JIT kompilací} --- Implementovat interaktivní
	      prostředí, které využívá just-in-time (JIT) kompilaci pro okamžité
	      vyhodnocování příkazů.

	\item \textbf{Standardní knihovna} --- Vytvořit základní knihovnu
	      obsahující běžně používané operace pro práci se zásobníkem,
	      aritmetiku, vstup a~výstup.
\end{enumerate}


\subsection{Struktura dokumentu}
\label{sec:struktura}

Dokument je strukturován následovně:

\textbf{Kapitola~\ref{sec:teorie}} poskytuje teoretický základ nutný
pro pochopení práce. Představuje jazyk \Forth{}, jeho historii
a~principy zásobníkové architektury. Dále se věnuje obecné teorii
kompilátorů, fázím překladu a~optimalizacím. Závěr kapitoly popisuje
relevantní vlastnosti jazyka Rust.

\textbf{Kapitola~\ref{sec:navrh}} se zabývá analýzou požadavků
na kompilátor a~návrhem jeho architektury. Definuje instrukční sadu
mezikódu a~navrhuje optimalizační průchody.

\textbf{Kapitola~\ref{sec:implementace}} detailně popisuje implementaci
jednotlivých částí kompilátoru: lexer, parser, sémantický analyzátor,
generátor mezikódu, optimalizátor, backendy pro generování kódu,
REPL prostředí a~standardní knihovnu.

\textbf{Kapitola~\ref{sec:testovani}} prezentuje metodiku testování
kompilátoru a~výsledky testů. Obsahuje také porovnání s~existujícími
implementacemi jazyka \Forth{}.

\textbf{Závěr} shrnuje dosažené výsledky, hodnotí splnění cílů
a~navrhuje možná rozšíření.

\textbf{Přílohy} obsahují elektronickou verzi zdrojových kódů,
uživatelskou dokumentaci a~formální gramatiku jazyka.
